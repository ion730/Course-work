\documentclass[14pt, oneside]{altsu-report}

\worktype{Шахматы на ЯП Python}
\title{Курсовая работа (2 курс)}
\author{А.\,А.~Медведев}
\groupnumber{5.205-2}
\GradebookNumber{1337}
\supervisor{И.\,А.~Шмаков}
\supervisordegree{ст. преп. каф. ВТиЭ}
\ministry{Министерство науки и высшего образования}
\country{Российской Федерации}
\fulluniversityname{ФГБОУ ВО Алтайский государственный университет}
\institute{Институт цифровых технологий, электроники и физики}
\department{Кафедра вычислительной техники и электроники}
\departmentchief{В.\,В.~Пашнев}
\departmentchiefdegree{к.ф.-м.н., доцент}
\shortdepartment{ВТиЭ}
\abstractRU{
Данная работа посвящена разработке игры «Шахматы» на языке программирования Python.

Целью курсовой работы является разработка функциональной интерактивной игры «Шахматы» на языке программирования Python с использованием библиотеки Tkinter, создание графического интерфейса программы, а также искусственного интеллекта соперника. В игру будут добавлены: возможность совместной игры, игры против искусственного интеллекта и таблица рекордов (самое быстрое окончание игры).
}
\keysRU{шахматы, игра, Python, Tkinter, программирование, искусственный интеллект, алгоритм}

\date{\the\year}

% Подключение файлов с библиотекой.
\addbibresource{graduate-students.bib}

% Пакет для отладки отступов.
%\usepackage{showframe}

\begin{document}
\maketitle

\setcounter{page}{2}
\makeabstract
\tableofcontents

\chapter*{Введение}
\phantomsection\addcontentsline{toc}{chapter}{ВВЕДЕНИЕ}

\textbf{Актуальность}
Актуальность данной работы обусловлена несколькими факторами:
\begin{enumerate}
    \item Python - универсальный язык программирования, широко используемый во многих областях
    \item Шахматы - одна из самых популярный игр в мире, которая развивает мышление, память, внимание и стратегиское планирование человека
\end{enumerate}

Данная работа позволит лучше понять работу языка Python и его базовой библиотеки Tkinter в частности.

Востребованность Python сложно переоценить - данный язык программирования используется практически везде, где задействовано программирование. Соответственно, умение владеть этим языком облегчит дальнейшую жизнь программиста.
\textbf{Цель}

Цель курсовой работы - создать игру "Шахматы на языке программирования Python с использованием библиотеки Tkinter, которая будет иметь следующие функции:
\begin{itemize}
    \item Возможность играть в шахматы с другим человеком на одном компьютере
    \item Возможность играть в шахматы с искусственным интеллектом
    \item Возможность отменять ходы
    \item Возможность настраивать внешний вид клеток
\end{itemize}
\textbf{Задачи:}
\begin{enumerate}
\item Изучить возможности библиотеки Tkinter
\item Реализовать клыссы фигур, варианты их ходов, проверки правил шахмат
\item Реализовать графический интерфейс игры с использованием виджетов, таких как окна, кнопки, меню
\item Протестировать работоспособность и удобство использования игры, а также исправить возможные ошибки 
\end{enumerate}
%\begin{comment}

\chapter{\label{ch:ch01}ГЛАВА 1} 
Книги по Python ~\cite{book1, book2}


\chapter{\label{ch:ch02}ГЛАВА 2}
Ещё книги - Tkinter ~\cite{book3, book4}

Сайт - Tkinter и Python ~\cite{site1}

%\include{chapter-3-report-csae.tex}
%\end{comment}
\chapter*{Заключение}
\phantomsection\addcontentsline{toc}{chapter}{ЗАКЛЮЧЕНИЕ}

По итогам работы была создана игра "Шахматы" с использованием языка программирования Python и встроенной библиотеки Tkinter. Была проведена проверка работоспособности программы, исправление ошибок.

\newpage
\phantomsection\addcontentsline{toc}{chapter}{СПИСОК ИСПОЛЬЗОВАННОЙ ЛИТЕРАТУРЫ}
\printbibliography[title={Список использованной литературы}]

\appendix
\newpage
\chapter*{\raggedleft\label{appendix1}Приложение}
\phantomsection\addcontentsline{toc}{chapter}{ПРИЛОЖЕНИЕ}
%\section*{\centering\label{code:appendix}Текст программы}

\begin{center}
\label{code:appendix}Заготовка для кода
\end{center}

\begin{code}
\captionof*{listing}{\centering\label{}Название программы}
\vspace{-1cm}\inputminted{C}{}
\end{code}

\end{document}

